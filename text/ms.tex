%%\documentclass[12pt, preprint]{aastex}
\documentclass[12pt, preprint]{emulateapj}
\usepackage{microtype}
\usepackage[breaklinks,colorlinks, urlcolor=blue,citecolor=blue,linkcolor=blue]{hyperref}
\usepackage{graphicx}	% For figures
\usepackage{natbib}	% For citep and citep
\usepackage{amsmath}	% for \iint
\usepackage{mathtools}
\usepackage{bm}
\usepackage[breaklinks]{hyperref}	% for blackboard bold numbers
\usepackage{hyperref}
\usepackage{algorithmic,algorithm}
\hypersetup{colorlinks}
\usepackage{color}
\usepackage{morefloats}
\definecolor{darkred}{rgb}{0.5,0,0}
\definecolor{darkgreen}{rgb}{0,0.5,0}
\definecolor{darkblue}{rgb}{0,0,0.5}
\hypersetup{ colorlinks,
linkcolor=darkblue,
filecolor=darkgreen,
urlcolor=darkred,
citecolor=darkblue}

\DeclareMathOperator*{\argmax}{arg\,max}

\newcommand{\todo}[1]{{\em \textcolor{red}{ #1}}}
\newcommand{\beq}{\begin{equation}}
\newcommand{\eeq}{\end{equation}}
\newcommand{\lang}{\langle}
\newcommand{\ra}{\rangle}
\newcommand{\vep}{\bm{\epsilon}}
\newcommand{\ep}{\epsilon}
\newcommand{\pars}{\vec{\theta}}
\newcommand{\dev}{\mathrm{d}}
\newcommand{\wpp}{w_{\rm p}}
\newcommand{\ngal}{n_{g}}
\newcommand{\gmf}{g(N)}
\newcommand{\mr}{M_{\rm r}}
\newcommand{\rpp}{r_{\rm p}}
\newcommand{\mzero}{\log M_{0}}
\newcommand{\mone}{\log M_{1}}
\newcommand{\mmin}{\log M_{\rm min}}
\newcommand{\sigmam}{\sigma_{\log \rm M}} 
\newcommand{\acen}{\mathcal{A}_{\rm cen}}
\newcommand{\asat}{\mathcal{A}_{\rm sat}}

\begin{document}

\author{Mohammadjavad Vakili\altaffilmark{1}, David~W.~Hogg\altaffilmark{1,2,3,4}, and ChangHoon~Hahn\altaffilmark{1}}
\affil{Center for Cosmology and Particle Physics, New York University, 4 Washington Pl, New York, NY, 10003}
\email{mjvakili@nyu.edu}

\altaffiltext{1}{Center for Cosmology and Particle Physics, New York University, 4 Washington Pl, New York, NY, 10003}
\altaffiltext{2}{Simons Center for Data Analysis, 160 Fifth Avenue, New York, NY}
\altaffiltext{3}{Center for Data Science, New York University, 726 Broadway, New York, NY}
\altaffiltext{4}{Max-Planck-Institut f\"ur Astronomie, K\"onigstuhl 17, D-69117 Heidelberg, Germany}
\begin{abstract}
Analyses of the $N$-body simulations have demonstrated that the clustering of dark matter halos depend on the halo properties beyond mass such as halo concentration. Standard halo occupation distribution (HOD) modeling approach in large scale structure studies assumes that the halo mass alone is sufficient in characterizing the galaxy-halo connection. 
Galaxy assembly bias hypothesizes that properties beyond halo mass influence the galaxy-halo connection, and can lead to systematic effects on the modeling of galaxy clustering.

Using the Small MultiDark Planck high resolution $N$- body simulation and the measurements of the projected two-point correlation function as well as the number density of the DR7 main galaxy sample of the Sloan Digital Sky Survey, we constrain the concentration-dependence of HOD. This concentration dependence is modeled within the framework of the proposed decorated HOD model. 

For fainter samples, we find non-zero albeit poor constraints on the strength of the assembly bias. Furthermore, we find that the assembly bias remains largely unconstrained in the brighter galaxy samples that live in massive dark matter halos.

By comparison between the 2PCF measured from the mock galaxy catalogs using our best-fit model and the randomized galaxy catalogs in which the assembly bias is erased, we note that the effect of assembly bias on galaxy clustering is qualitatively similar to its effect on galaxy catalogs created from a certain class of subhalo abundance matching models. 
\end{abstract}
\keywords{Cosmology: large-scale-structure, galaxy: galaxies}
\end{abstract}

\section{Introduction}
The assumption that galaxies reside dark matter halos 
is the central tenet of the modern cosmology and 
large-scale structure formation theories. The study of galaxy-halo connection is an essential tool for constraining the cosmology via small scale clustering and galaxy formation physics. One of the most powerful methods for describing the galaxy-halo connection is the halo occupation distribution (HOD, see \citealt{seljak2000, berlind_weinberg2002, scoccimarro2001, zheng2005, zheng2007}) and conditional luminosity function (CLF, see \citealt{clf_vdBosch}).

HOD provides an analytical prescription for the statistical properties of galaxies within dark matter halos by specifying a probability distribution function $P(N|x)$ where $N$ is the number of galaxies and $x$ is a property of the dark matter halo. The underlying assumption of the standard HOD framework is 
that the halo mass alone is sufficient in determining the
population of galaxies inside halo. That is, the statistical properties of 
galaxies is governed by the halo mass. Mathematically, this assumption can be 
written as $P(N_g|M_h,\{x\})=P(N_g|M_h)$ where $\{x\}$ is the set of all 
possible halo properties beyond mass $M_{h}$. In a similar fashion, CLF ties the luminosity of galaxies to the host halo mass.

Despite this very simplifying assumption, the galaxy-halo connection models based on HOD and CLF have been used in fitting the measurements of a wide range of statistics such as correlation function of galaxies and galaxy-galaxy lensing. HOD has been used in constraining the cosmological parameters through modeling the galaxy two-point correlation function (hereafter 2PCF) (\citealt{abazajian2005}), combination of 2PCF with mass-to-light ratio of galaxies (\citealt{tinker05}), redshift space distortions (\citealt{tinker_rsd2007}), mass-to-number ratio of galaxy clusters (\citealt{tinker2012}) galaxy-galaxy lensing in SDSS main sample of galaxies (\citealt{vdb03,cacciato13,more13,vdb13}), and galaxy-galaxy lensing in the BOSS galaxies (\citealt{miyatake15,more15}). It has also been used in galaxy evolution studies (\citealt{conroy09,leauthaud12,behroozi13}).

It has been shown by the $N$-body simulations that the halo clustering is correlated with the formation history of halos. That is, at a fixed halo mass, the halo bias is correlated with properties of halos beyond mass such as the concentration, formation time, spin, and etc. This phenomenon is known as the halo assembly bias (see \citealt{sheth2004,gao2005, harker2006, weschler2006, gao2007,croton2007,wang2007,angulo2008,dalal2008,li2008,sunayama2016}). From the observational standpoint, this phenomenon has been confirmed
(\citealt{miyatake2016,more2016}). 

If the galaxy properties are tied to the formation history of their host halos, we expect the occupation statistics of galaxies to be coupled to the properties as the concentration of halos, environment, and etc. This is called galaxy assembly bias. In the context of the semi-analytic models of galaxy formation, \citealt{croton2007} shows that 

However, the assumption that the statistical 
distribution of galaxies can be determined by halo mass is not a derived characteristic of accurate and high resolution hydrodynamic simulations, but rather a theoretical assumption that needs to be scrutinized. In recent years, the idea that the   

In this investigation, we explore the notion that how the dependence of 
galaxy-halo connection on halo properties beyond mass (e.g. concentration, age).
Throughout this paper, unless stated otherwise, 
all radii and densities are in comoving units. Standard flat $\Lambda$CDM is assumed, 
and all cosmological parameters are set to the Planck 2013 best-fit estimates.

\section{Method}
\subsection{Halo occupation modeling}
\subsubsection{standard HOD modeling}\label{subsubsec:hod}
For our standard HOD modeling, HOD parameterization of cite[zheng07] is assumed. According to this parameterization, the occupation of the central galaxies follows a nearest-integer distribution, 
and the occupation of the satellite galaxies follows a Poisson distribution. The expected number of centrals and satellites as a function of the host halo mass of $M_{\rm h}$ are given by the following equations 
\begin{eqnarray}
\langle N_{\rm cen}(M_{\rm h}) \rangle &=& \frac{1}{2}\Big[1+\Big(\frac{\log M_{\rm h} - \log M_{\rm min}}{\sigma_{\log \rm{M}}} \Big) \Big], \label{hod:central}\\ 
\langle N_{\rm sat}(M_{\rm h}) \rangle &=& \Big( \frac{M_{\rm h} - M_{\rm{0}}}{M_{\rm 1}} \Big)^{\alpha}. \label{hod:satellite}
\end{eqnarray}

For populating the halos with galaxies, we follow the procedure described in \citealt{decorated}, \todo{HAHN+VAKILI 16 preparation}. The central galaxies are assumed to be at the center of the host dark matter halos. We place the satellite galaxies are within the virial radius of the halo following an NFW profile (\citealt{nfw}). The concentration of the NFW profile is given by the empirical mass-concentration relation provided by (\citealt{nfw_c(M)}). The velocities of the satellite galaxies are given by two components. The first component is the velocity of the host halo. The second component is the velocity of the satellite galaxy with respect to the host halo which is computed following the solution to the NFW profile Jeans equations (\citealt{more2010}). We refer the readers to \citealt{decorated} for a more comprehensive and detailed discussion of the forward modeling of the galaxy mock catalogs.

As described in \citealt{decorated} and \todo{HAHN+VAKILI in prep}, this approach makes no appeal to the fitting functions used in the analytical calculation of the 2PCF. The accuracy of these fitting functions is limited (\citealt{tinker08,tinker10,watson13}). This approach also does not face the known issues of properly treatment of halo exclusion and scale-dependent bias that can lead to significant inaccuracies (\citealt{vdb13}). 

Another added advantage of this forward modeling approach is that it permits us to take advantage of observables beyond the traditionally used 2PCF. One of these observables is the group multiplicity function which characterizes the abundance of galaxy groups. There is no fitting function for calculation of this observable, and therefore, our approach of direct population of dark matter halos allows us to use these unconventional measurements for constraining the parameters governing the galaxy-halo collection. 

\subsubsection{Assembly bias HOD modeling}\label{subsubsec:decorated}
Now let us provide a brief overview of HOD modeling decorated with $\mathtt{Heaviside}$ $\mathtt{Assemblybias}$ introduced in \citealt{decorated}. At a fixed halo mass $M_{\rm h}$, halos are split into two populations: population of halos with the 0.5-percentile of highest concentration, and population of halos with 0.5 percentile of lowest concentration.We call the first population "type-1" halos, and the second population "type-2 halos. In the decorated HOD model, the expected number of central and satellite galaxies in the two populations is given by

\begin{eqnarray}
\langle N_{c,i} | M_{h},c\rangle &=& \langle N_{c} | M_{h}\rangle + \delta N _{c,i}, \; i=1,2 \label{eq:decoratedcentral} \\
\langle N_{s,i} | M_{h},c\rangle &=& \langle N_{s} | M_{h}\rangle + \delta N _{s,i}, \; i=1,2 \label{eq:decoratedsatellite}
\end{eqnarray}
where $\langle N_{c} | M_{h}\rangle$ and $\langle N_{s} | M_{h}\rangle$ are given by Eqs \ref{hod:central} and \ref{hod:satellite} respectively, and we have $\delta N_{s,0} + \delta N_{s,1} = 0$, and $\delta N_{c,0} + \delta N_{c,1} = 0$. These two conditions ensure the conservation of HOD. At a given host hale mass $M_{h}$, the central occupation of the the two populations follows a Nearest-integer distribution with the first moment given by \ref{eq:decoratedcentral}; and the central occupation of the the two populations follows a Poisson distribution with the first moment given by \ref{eq:decoratedsatellite}. 

In this decorated occupation model, the allowable range for the quantities $\delta N_{c,1}$ and $\delta N_{s,1}$ is given by 
\begin{eqnarray}
-\langle N_{c} | M_{h}\rangle \leq \delta N_{c} \leq \langle N_{c} | M_{h}\rangle
 , \label{eq:cen-bounds} \\
-\langle N_{s} | M_{h}\rangle \leq \delta N_{s} \leq \langle N_{s} | M_{h}\rangle. \label{eq:sat-bounds}
\end{eqnarray}
Afterwards, the assembly bias parameter $\mathcal{A}$ is defined in the following way:
\begin{eqnarray}
\delta N_{\alpha , 1}(M_{h}) &=& \mathcal{A_\alpha} \delta N_{\alpha , 1}^{\rm max}(M_{h}) \; \; \rm{if} \; \mathcal{A_\alpha} > 0,  \\
\delta N_{\alpha , 1}(M_{h}) &=& \mathcal{A_\alpha} \delta N_{\alpha , 1}^{\rm min}(M_{h}) \; \; \rm{if} \; \mathcal{A_\alpha} < 0,
\end{eqnarray}
where the subscript $\alpha = c , s$ for the centrals and satellites respectively, and $\delta N_{\alpha , 1}^{\rm max}(M_{h})$, $\delta N_{\alpha , 1}^{\rm min}(M_{h})$ are given by Eqs. \ref{eq:cen-bounds} and \ref{eq:sat-bounds}. 

At a given halo mass $M_{h}$, once the first moments of occupation statistics for the $type$-1 and $type$-2 halos are determined, the same procedure described in \ref{subsubsec:hod} is done to populate the halos with mock galaxies.

\subsubsection{Redshift-space distortion}

Once the halo catalogs are populated with galaxies, the real-space positions and velocities of all mock galaxies are obtained. The next step is applying a redshift-space distortion transformation. Assuming the $\hat{z}$ axis for the direction of the RSD, the transformation $(x,y,z) \rightarrow (s_x,s_y,s_z) = (x , y ,z + v_z/cH_{0})$ is done to obtain the redshift-space coordinate of the produced mock galaxies.   

\subsection{Model Observables}
\subsubsection{group multiplicity function}
Group multiplicity function $g(N)$ is defined as the abundance of the galaxy groups as a 
function of group richness (the number of galaxies in groups) per richness width, measured in bins of richness. 
Galaxy groups in are found by Friends-of-Friends (hereafter FoF) group finder algorithm (\citealt{fof}). 
The FoF group finder algorithm links to galaxy $i$ and galaxy $j$ to each other, if the projected distance between the two galaxies $D_{\perp , i,j}$ is less than a specified projected linking length $b{_\perp}$, and the line-of-sight distance between the two galaxies $D_{\parallel, i,j}$ is less than a specified line-of-sight linking length $b{_\parallel}$. These linking lengths are measured in units of the mean inter-particle distance ($n_{\rm g}^{-1/3}$). In order to be consistent with the \citealt{berlind2006} measurements described in section \ref{sec:data}, the linking lengths $b_{\perp}$ and $b_{\parallel}$ are set to to 0.15 and 0.74 respectively. 

After populating the dark matter halo catalogs with galaxies, we use the $\mathtt{Halotools}$ implementation of the FoF group-finder algorithm to identify the FoF groups with the mentioned linking lengths. Afterwards, we compute the number density of the galaxy groups in bins of richness to measure the group multiplicity function. Note that $g(N)$ is measured in units of $h^{3}\rm{Mpc}^{-3}$.
\subsubsection{projected two-point correlation function}
The projected two point correlation function $w_{\rm{p}}(r_{\rm{p}})$ can be computed by integrating the 3D redshift space 2PCF $\xi(r_{\rm{p}} , \pi)$ along the line-of-sight
\beq
w_{\rm{p}}(r_{\rm{p}}) = 2 \int_{0}^{\pi_{max}} d\pi \;\xi(r_{\rm{p}} , \pi)
\eeq
For our 2PCF calculations, we use the $w_{\rm p}$ measurement functionality of the fast and publicly available pair-counter code $\mathtt{CorrFunc}$ (\citealt{corrfunc}).  To be consistent with the SDSS measurements described in \ref{sec:data}, $w_{\rm{p}}(r_{\rm{p}})$ is obtained by the line-of-sight integration to $\pi_{max}=40 \; h^{-1}\rm{Mpc}$. Note that $w_{\rm{p}}(r_{\rm{p}})$ is measured in units of $h^{-1}\rm{Mpc}$.

\subsection{Simulation}

For the simulations used in this work, we make use of the Rockstar (\citealt{rockstar}) halo catalogs in the $z=0$ Bolshoi-Planck high 
resolution $N$-body simulation (\citealt{bolshoi}). This simulation was carried out using the adaptive refinement tree code 
(ART, see \citealt{art}) code, following the Planck $\Lambda$CDM cosmological parameters 
$\Omega_{\rm m} = 0.307$, $\Omega_{\rm b} = 0.048$, $\Omega_{\Lambda} = 0.693$, $\sigma_{8} = 0.823$, $n_{\rm s}=0.96$, 
$h=0.678$. The Box size for this $N$-body simulation is 250 $h^{-1} \rm{Mpc}$, the number of simulation particles is 2048$^3$, 
the mass per simulation particle $m_{\rm p}$ is $1.5 \times 10^{8} \; h^{-1} M_{\odot}$, and the gravitational softening length 
$\epsilon$ is 1 $h^{-1} \rm{kpc}$. 

\section{Data}\label{sec:data}

We focus on two sets of measurements made on the volume limited luminosity threshold main sample of galaxies in the SDSS spectroscopic survey. In this section, we briefly describe the measurements used in our study for finding constraints on the assembly bias parameters.

\subsection{group multiplicity function}

The first set of measurements consists of the galaxy number densities $n_{g}$ and the group multiplicity functions $g(N)$ described in \citealt{berlind2006}. These measurements are made in three Volume-limited samples of the SDSS galaxies with luminosity thresholds of $M_{r}<-18$, $M_{r}<-19$, and $M_{r}<-20$. These galaxy groups are found by Friends-of-Friends (hereafter FoF) group finder algorithm (\citealt{fof}). \citealt{berlind2006} sets $b_{\perp}$ to 0.15 and $b_{\parallel}$ to 0.74 to minimize the discrepancy between the multiplicity function, projected size, and the number of FoF groups and those of the halos in a suite of $N$-body simulations.

The measurements are accompanied by two sets of uncertainty measurements. The first one is the error on $g(N)$ that stems from the dispersion between $g(N)$ measurements of different mock catalogs. The second source of error is the Poisson error that results from the number of groups in each richness bin. Finally, these two uncertainties are added in quadrature 
to estimate the total error on these $g(N)$ measurements.

\subsection{projected 2PCF}

The second set consists of the number density $n_{g}$ and the projected 2PCF $w_{\rm{p}}(r_{\rm{p}})$ made by \citealt{guo2015} for Volume-limited sample of galaxies in SDSS DR7 galaxy catalog. The project 2PCFs are measured in 12 logarithmic $r_{\rm p}$ bins (in units of $h^{-1}\rm{Mpc}$) of width $\Delta \log(r_{\rm p}) = 0.2$, starting from $r_{\rm{p}} = 0.1 \; h^{-1}\rm{Mpc}$. These measurements are accompanied by their corresponding covariance matrices constructed using 400 jackknife sub-samples of the data. The advantage of using these measurements is that the effects of fiber collision systematics on the two-point statistics are corrected for, and therefore, these measurements provide very accurate small scale clustering measurements (\citealt{guo2012,guo2015}). As \citealt{hearin15} demonstrates, the satellite assembly bias $\mathcal{A}_{\rm sat}$ can lead to 10 $\%$ level impact on the small scale galaxy clustering.  

\section{Analysis and Results}

\subsection{Inference setup}

We aim to constrain the parameters of the HOD model described in \ref{subsubsec:hod}, and the parameters of the assembly bias HOD model described in \ref{subsubsec:decorated}, given two sets of SDSS measurements described in \ref{sec:data}. For the tests with group multiplicity function as observable, we use measurements with maximum absolute luminosity $M_{\rm r , max}$ of -18, -19, and -20. Finally, for the tests with the projected 2PCF, we use measurements with $M_{\rm r , max}$ of -18, -19, -20, and -21.  

We sample from the posterior probability distribution $p(\theta|d) \propto p(d|\theta) \pi(\theta)$. In the standard HOD modeling 
\beq
\theta = \{ \mmin, \; \sigmam, \; \mzero, \; \alpha, \; \mone \},
\eeq
and in the Heaviside assembly bias modeling we have 
\beq
\theta = \{ \mmin, \; \sigmam, \; \mzero, \; \alpha, \; \mone, \; \acen, \; \asat \},
\eeq
Furthermore, data (denoted by $d$) is the combination $[n_{g},g(N)]$ when GMF is used, and $[n_{g}, w_{\rm{p}}(r_{\rm{p}})]$ when 2PCF is used.

Now we need to specify the likelihood function $p(d|\theta)$. In the case where the observables are the number density and the GMF are used the negative log-likelihood is given by
\begin{eqnarray}
-2\ln p(d|\theta) &=& \frac{[n^{\rm data}_{g}-n^{\rm model}_{g}]^{2}}{\sigma_{n}^{2}} \nonumber \\ &+& \sum_{N} \frac{[ g^{\rm data}(N)-g^{\rm model}(N)]^{2}}{\sigma^{2}_{N}} \; + \; \rm{const.},
\label{eq:lnlike_group}
\end{eqnarray}
where the summation is over the group richness bins, and $\sigma_{N}$ is the uncertainty associated with the richness bin $N$. In the case, where the number density and 2PCF are used, the negative log-likelihood is given by
\begin{eqnarray}
-2\ln p(d|\theta) &=& \frac{[n^{\rm data}_{g}-n^{\rm model}_{g}]^{2}}{\sigma_{n}^{2}} \nonumber \\  
&+&  \sum_{\rm p} \Delta \wpp^{\rm T}C^{-1}\Delta \wpp \; + \; \rm{const.},
\label{eq:lnlike_wp}
\end{eqnarray}
where $\Delta \wpp(\rpp) = \wpp^{\rm data}(\rpp)-\wpp^{\rm model}(\rpp)$, and  $C^{-1}$ is inverse of the estimate of the jackknife covariance matrix provided by \citealt{guo15}. 
Another important ingredient of our analysis is specification of the prior probabilities over the parameters of the two halo occupation models. For both model, we use uniform flat priors for all the parameters. The prior ranges are specified in the Table \ref{tab:prior}. 

\begin{table*}
\begin{center}
  \label{tab:prior}
  \caption{{\bf Prior Specifications}: The prior probability distribution 
  and its range for each of the parameters. 
  All mass parameters are in unit of $h^{-1}M_\odot$. The parameters marked by $*$ are only used in the Heaviside Assembly bias modeling.}
\begin{tabular}{@{}lllll}
\\ \hline 
    Parameter & & Prior & & Range \\ \hline
  $\alpha$ & & Uniform & & [0.85, 1.45] \\
  $\sigmam$ & & Uniform & &  [0.05, 1.5] \\
   $\mzero$   & & Uniform & &  [10.0, 14.5] \\
  $\mmin$ & &   Uniform & &  [10.0, 14.0] \\
  $\mone$ & & Uniform & & [11.5, 15.0] \\ 
  $\acen^{*}$ & & Uniform & & [-1.0, 1.0] \\
  $\asat^{*}$ & & Uniform & & [-1.0, 1.0] \\
 \hline
  \end{tabular}
\end{center}
\end{table*}

\begin{table*}
\begin{center}
  \label{tab:prior}
  \caption{{\bf Constraints}: The prior probability distribution 
  and its range for each of the parameters. 
  All mass parameters are in unit of $h^{-1}M_\odot$. The first five rows are the only parameters of the standard HOD modeling.}
\begin{tabular}{@{}lllllllllllllllllllllllll}
\\ \hline 
    Model & Data & $\mr$ & $\mmin$ & $\sigmam$ & $\mzero$ & $\alpha$ &  $\mone$ & $\acen$ & $\asat$ & $\chi^{2}/\rm{dof}$ & $\rm{AIC}$ & $\rm{BIC}$\\  \hline
  \\
  HOD & $\gmf$, $\bar{n}$ & 18 & $11.58^{+0.27}_{-0.22}$ &  $0.75^{+0.51}_{-0.48}$ & $10.93^{+0.68}_{-0.64}$ & $0.94^{+0.10}_{-0.07}$ &  $12.46^{+0.10}_{-0.09}$ & $-$ & $-$ & 5.55/1 & 21.55 & 19.42\\ \\
  
  AB & $\gmf$, $\bar{n}$ & 18 & $11.57^{+0.26}_{-0.20}$ &  $0.76^{+0.50}_{-0.50}$ & $10.83^{+0.65}_{-0.57}$ & $0.94^{+0.10}_{-0.06}$ &  $12.47^{+0.12}_{-0.09}$ & $0.13^{+0.63}_{-0.75}$ & $-0.01^{+0.60}_{-0.59}$ & 6.46/1 & 34.46 & 25.87\\ \\
   
HOD & $\gmf$, $\bar{n}$ & 19 & $11.87^{+0.32}_{-0.23}$ &  $0.70^{+0.53}_{-0.47}$ & $10.93^{+0.74}_{-0.65}$ & $0.99^{+0.10}_{-0.08}$ &  $12.73^{+0.07}_{-0.07}$ & $-$ & $-$ & 5.19/1 & 18.72 & 20.87\\ \\
    
AB & $\gmf$, $\bar{n}$ & 19 & $11.88^{+0.31}_{-0.23}$ &  $0.74^{+0.50}_{-0.47}$ & $10.90^{+0.75}_{-0.62}$ & $0.97^{+0.10}_{-0.08}$ &  $12.73^{+0.07}_{-0.07}$ & $0.13^{+0.62}_{-0.73}$ & $-0.03^{+0.63}_{-0.59}$ & 6.66/1 & 28.13 & 28.61\\ \\

HOD  & $\gmf$, $\bar{n}$ & 20 & $12.16^{+0.36}_{-0.20}$ &  $0.58^{+0.50}_{-0.36}$ & $10.94^{+0.80}_{-0.66}$ & $0.98^{+0.09}_{-0.08}$ &  $13.07^{+0.07}_{-0.06}$ & $-$ & $-$ & 5.16/1 & 18.32 & 21.26\\ \\

AB  & $\gmf$, $\bar{n}$ & 20 & $12.14^{+0.34}_{-0.19}$ &  $0.56^{+0.47}_{-0.35}$ & $10.95^{+0.72}_{-0.64}$ & $0.96^{+0.10}_{-0.07}$ &  $13.08^{+0.07}_{-0.06}$ & $0.04^{+0.68}_{-0.70}$ & $-0.04^{+0.63}_{-0.58}$ & 6.04/1 & 26.63 & 28.57\\ \\

%%%%%%%wp%%%%%

HOD  & $\wpp$, $\bar{n}$ & 18 & $11.46^{+0.16}_{-0.17}$ &  $1.11^{+0.28}_{-0.47}$ & $10.85^{+0.74}_{-0.59}$ & $1.03^{+0.05}_{-0.06}$ &  $12.54^{+0.10}_{-0.10}$ & $-$ & $-$ & 31.72/13 & 50.29 & 44.55\\ \\

AB & $\wpp$, $\bar{n}$ & 18 & $11.44^{+0.15}_{-0.15}$ &  $1.20^{+0.21}_{-0.35}$ & $10.91^{+0.79}_{-0.64}$ & $1.02^{+0.06}_{-0.06}$ &  $12.61^{+0.11}_{-0.11}$ & $-0.82^{+0.27}_{-0.14}$ & $0.19^{+0.67}_{-0.73}$ & 31.81/13 & .55 & 19.42 \\ \\

HOD & $\wpp$, $\bar{n}$ & 18.5 & $11.64^{+0.23}_{-0.24}$ &  $0.89^{+0.43}_{-0.55}$ & $10.75^{+0.65}_{-0.51}$ & $1.06^{+0.05}_{-0.05}$ &  $12.72^{+0.09}_{-0.08}$ & $-$ & $0.37^{+0.44}_{-0.52}$ & 18.11/13 & 36.69 & 30.94\\ \\

AB & $\wpp$, $\bar{n}$ & 18.5 & $11.59^{+0.24}_{-0.21}$ &  $0.85^{+0.44}_{-0.54}$ & $10.77^{+0.63}_{-0.52}$ & $1.06^{+0.05}_{-0.06}$ &  $12.78^{+0.10}_{-0.10}$ & $-0.14^{+0.69}_{-0.56}$ & $0.37^{+0.44}_{-0.52}$ & 19.53/13 & 55.93 & 37.48\\ \\

HOD & $\wpp$, $\bar{n}$ & 19 & $11.84^{+0.28}_{-0.27}$ &  $0.89^{+0.41}_{-0.51}$ & $10.79^{+0.65}_{-0.54}$ & $1.08^{+0.04}_{-0.04}$ &  $12.89^{+0.09}_{-0.09}$ & $-$ & $-$ & 24.20/13 & 42.77 & 37.03\\ \\

AB & $\wpp$, $\bar{n}$ & 19 & $11.78^{+0.30}_{-0.24}$ &  $0.81^{+0.48}_{-0.51}$ & $10.77^{+0.66}_{-0.53}$ & $1.07^{+0.04}_{-0.05}$ &  $12.95^{+0.10}_{-0.10}$ & $0.15^{+0.52}_{-0.61}$ & $0.52^{+0.33}_{-0.44}$ & 22.80/13 & 59.20 & 40.75\\ \\

HOD & $\wpp$, $\bar{n}$ & 19.5 & $11.85^{+0.31}_{-0.19}$ &  $0.66^{+0.46}_{-0.40}$ & $10.88^{+0.67}_{-0.59}$ & $1.13^{+0.04}_{-0.04}$ &  $13.07^{+0.07}_{-0.07}$ & $-$ & $-$ & 15.89/13 & 34.46 & 28.72\\ \\

AB & $\wpp$, $\bar{n}$ & 19.5 & $11.82^{+0.38}_{-0.18}$ &  $0.63^{+0.55}_{-0.40}$ & $10.83^{+0.74}_{-0.57}$ & $1.10^{+0.04}_{-0.05}$ &  $13.07^{+0.08}_{-0.08}$ & $0.35^{+0.40}_{-0.52}$ & $0.03^{+0.59}_{-0.41}$ & 23.17/13 & 37.17 & 41.12\\ \\

HOD & $\wpp$, $\bar{n}$ & 20 & $11.99^{+0.15}_{-0.08}$ &  $0.36^{+0.28}_{-0.21}$ & $11.05^{+0.76}_{-0.71}$ & $1.17^{+0.04}_{-0.04}$ &  $13.34^{+0.07}_{-0.07}$ & $-$ & $-$ & 18.71/13 & 37.28 & 31.54\\ \\

AB & $\wpp$, $\bar{n}$ & 20 & $12.13^{+0.39}_{-0.18}$ &  $0.65^{+0.47}_{-0.38}$ & $11.14^{+0.80}_{-0.77}$ & $1.11^{+0.06}_{-0.06}$ &  $13.31^{+0.09}_{-0.09}$ & $0.71^{+0.21}_{-0.41}$ & $0.02^{+0.49}_{-0.33}$ & 15.22/13 & 51.62 & 33.17\\ \\
                 
 \hline
  \end{tabular}
\end{center}
\end{table*}


For sampling from the posterior probability, given the likelihood function (see Eqs. \ref{eq:lnlike_group}, \ref{eq:lnlike_wp}) and the prior probability distributions (see Table \ref{tab:prior}), we use the affine-invariant ensemble MCMC sampler (\citealt{goodmanweare}) and its implementation $\mathtt{emcee}$ (\citealt{emcee}). In particular, we run the $\mathtt{emcee}$ code with 140 walkers and we run the chains for at least 15000 iterations until the convergence of the MCMC chains is ensured. For testing the convergence of the chains, we perform Gelman-Rubin convergence test (\citealt{grtest}).

\subsection{results}


\subsection{model comparison}

Now that we have presented constraints on the parameters of the Heaviside assembly bias HOD model and those found from the standard HOD model using the group and clustering measurements of the SDSS galaxies, it is necessary to make comparison between the the two models in terms of the goodness of fit. Evidently, the Heaviside Assembly bias model has more complexity for modeling the halo occupation. The question that arises is whether this higher level of model complexity is more preferable by the group statistics and the clustering data used in this study. 

There are various methods for model comparison. In this work we make use of two model comparison criteria: \emph{Akaike Information Criterion} (AIC, \citealt{aic} , see \citealt{gelmanic} for detailed discussion in AIC), and \emph{Bayesian Information Criteria} (BIC, \citealt{bic}). AIC and BIC are two widely used methods for assessing the relative quality of models given a set of observations.
Suppose that $\mathcal{L}_{\rm max}$ is the maximum likelihood of the data given a model, $N_{\rm par}$ is the number of parameters of a given model, and $N_{\rm data}$ is the number of data points. Then we have

\begin{eqnarray}
\rm{AIC}&=& -2\;\ln \mathcal{L}_{\rm max} + 2N_{\rm par} \nonumber \\
         &+& 2\;\frac{N_{\rm par}(N_{\rm par}+1)}{N_{\rm data}-N_{\rm par}-1} \; , \label{eq:aic}\\
\rm{BIC}&=& -2\;\ln \mathcal{L}_{\rm max} + N_{\rm par}\;\ln N_{\rm data} \; . \label{eq:bic}
\end{eqnarray}
Models that deliver lower values of information criteria are more favorable. In Eqs. (\ref{eq:aic}) and (\ref{eq:bic}) $N_{\rm par}$ is 5 when the standard HOD model is used, and 7 when the Heaviside assembly bias model is used. When the observables are $\ngal$ and $\wpp$, $N_{\rm data}$ is 13 for all luminosity thresholds, and when the observables are $\ngal$ and $\gmf$, $N_{\rm data}$ is 16 for the $\mr < -18$ luminosity threshold, 23 for the $\mr < -19$ luminosity threshold, and 25 for the $\mr < -20$ luminosity threshold. 




\section{Discussion}



%%%%%%%%%%%FIGURES%%%%%%%%%%%%%%%

%%%%%%%%%%%%%%%%%%%%%%%%%%%%%%%%%%%%%%%%%%%%%%%%%%%%%%%%
% gmf-Mr18-posterior 
%%%%%%%%%%%%%%%%%%%%%%%%%%%%%%%%%%%%%%%%%%%%%%%%%%%%%%%%
\begin{figure*}
\begin{center}
\includegraphics[width=\textwidth]{post18_0gmf.pdf}
\caption{We present the constraints on the Heaviside assembly bias HOD model parameters (blue) and the standard HOD model parameters (yellow) obtained from our analysis using $\bar{n}$ and $\gmf$ measurements of galaxies with $\mr < -18$. The diagonal panels plot the marginalized posterior distribution of each parameter with vertical dashed lines marking the $50\%$ quantile and of the distribution. The off-diagonal panels plot the degeneracies between parameter pairs. The range of each panel corresponds to the range of our prior choice. The parameters $\acen$ and $\asat$ are only defined for the Assembly bias model. For both models, the parameter $\sigmam$ remains unconstrained. In The assembly bias model, $\acen$ and $\asat$ remain largely unconstrained. The parameter $\alpha$ does not have a  lower bound constraint in both models. The constraints on the parameters $\mzero$, $\mmin$, $\alpha$, $\mone$ found from the two models are in strong agreement.}
\label{fig:post_gmf_18}
\end{center}
\end{figure*}

%%%%%%%%%%%%%%%%%%%%%%%%%%%%%%%%%%%%%%%%%%%%%%%%%%%%%%%%
% gmf-Mr19-posterior 
%%%%%%%%%%%%%%%%%%%%%%%%%%%%%%%%%%%%%%%%%%%%%%%%%%%%%%%%
\begin{figure*}
\begin{center}
\includegraphics[width=\textwidth]{post19_0gmf.pdf}
\caption{Same as Figure \ref{fig:post_gmf_18} but for $\bar{n}$ and $\gmf$ measurements of galaxies with $\mr < -19$. Again, For both models, the parameter $\sigmam$ remains unconstrained. In The assembly bias model, $\acen$ and $\asat$ remain largely unconstrained. The  marginalized posterior pdf over the parameter $\alpha$ found from the standard HOD model is peaked at a slightly higher value of $\alpha$ than the one found from the assembly bias HOD model. The constraints on the parameters $\mzero$, $\mmin$, and $\mone$ found from the two models are in strong agreement.}
\label{fig:post_gmf_19}
\end{center}
\end{figure*}

%%%%%%%%%%%%%%%%%%%%%%%%%%%%%%%%%%%%%%%%%%%%%%%%%%%%%%%%
% gmf-Mr20-posterior 
%%%%%%%%%%%%%%%%%%%%%%%%%%%%%%%%%%%%%%%%%%%%%%%%%%%%%%%%
\begin{figure*}
\begin{center}
\includegraphics[width=\textwidth]{post20_0gmf.pdf}
\caption{Same as Figure \ref{fig:post_gmf_20} but for $\bar{n}$ and $\gmf$ measurements of galaxies with $\mr < -20$. In The assembly bias model, $\acen$ and $\asat$ remain largely unconstrained. The  marginalized posterior pdf over the parameter $\alpha$ found from the standard HOD model is peaked at a slightly higher value of $\alpha$ than the one found from the assembly bias HOD model. The constraints on the parameters $\sigmam$, $\mzero$, $\mmin$, and $\mone$ found from the two models are in strong agreement.}
\label{fig:post_gmf_20}
\end{center}
\end{figure*}

%%%%%%%%%%%%%%%%%%%%%%%%%%%%%%%%%%%%%%%%%%%%%%%%%%%%%%%%
% wp-Mr18-posterior 
%%%%%%%%%%%%%%%%%%%%%%%%%%%%%%%%%%%%%%%%%%%%%%%%%%%%%%%%
\begin{figure*}
\begin{center}
\includegraphics[width=\textwidth]{post18_0wp.pdf}
\caption{We present the constraints on the Heaviside assembly bias HOD model parameters (blue) and the standard HOD model parameters (yellow) obtained from our analysis using $\bar{n}$ and $\wpp$ measurements of galaxies with $\mr < -18$. The diagonal panels plot the marginalized posterior distribution of each parameter with vertical dashed lines marking the $50\%$ quantile and of the distribution. The off-diagonal panels plot the degeneracies between parameter pairs. The range of each panel corresponds to the range of our prior choice. The parameters $\acen$ and $\asat$ are only defined for the Assembly bias model. For both models, the parameter $\sigmam$ remains unconstrained. In The assembly bias model, $\asat$ remains largely unconstrained. The constraint on the parameter $\acen$ however, indicates that negative values of this parameter are strongly favored. Specifically the best-fit value of $\acen$ is -0.82 with its 68$\%$ confidence interval being less than -0.55. The constraints on $\mzero$ and $\mmin$ found from the two models strongly agree. The constraints on $\alpha$ and $\mone$ found from the two model in overall agreement with a slightly higher value of $\alpha$ and a slightly lower value of $\mone$ in the standard HOD model.}
\label{fig:post_wp_18}
\end{center}
\end{figure*}

%%%%%%%%%%%%%%%%%%%%%%%%%%%%%%%%%%%%%%%%%%%%%%%%%%%%%%%%
% wp-Mr19-posterior 
%%%%%%%%%%%%%%%%%%%%%%%%%%%%%%%%%%%%%%%%%%%%%%%%%%%%%%%%
\begin{figure*}
\begin{center}
\includegraphics[width=\textwidth]{post19_0wp.pdf}
\caption{Same as Figure \ref{fig:post_wp_18} but for $\bar{n}$ and $\gmf$ measurements of galaxies with $\mr < -19$. Once again, the parameter $\sigmam$ remains unconstrained. $\acen$ is not constrained in the assembly bias model, but constraints on $\asat$ show that positive values of $\asat$ are strongly favored. The best-fit value of $\asat$ is 0.52 and its 68$\%$ confidence interval is larger than 0.08. The constraints on $\mzero$, $\mmin$, $\alpha$, and $\mone$ found from the two models are in good agreement, with $\alpha$ being slightly higher and $\mone$ being slightly lower in the standard HOD model.}
\label{fig:post_wp_19}
\end{center}
\end{figure*}

%%%%%%%%%%%%%%%%%%%%%%%%%%%%%%%%%%%%%%%%%%%%%%%%%%%%%%%%
% wp-Mr20-posterior 
%%%%%%%%%%%%%%%%%%%%%%%%%%%%%%%%%%%%%%%%%%%%%%%%%%%%%%%%
\begin{figure*}
\begin{center}
\includegraphics[width=\textwidth]{post20_0wp.pdf}
\caption{Same as Figure \ref{fig:post_wp_18} but for $\bar{n}$ and $\gmf$ measurements of galaxies with $\mr < -20$. For this sample of galaxies, the parameter $\sigmam$ is better constrained by the two models, with the constraints coming from the standard HOD model being tighter. $\asat$ is not constrained in the assembly bias model and is consistent with zero, but constraints on $\acen$ show that positive values of $\acen$ are strongly favored. The best-fit value of $\acen$ is 0.71 and its 68$\%$ confidence interval is larger than 0.30. The constraints on $\mzero$ and $\mone$ found from the two models are in good agreement. There is some discrepancy between the constraints on $\mmin$ and $\alpha$ from the standard HOD model and  the ones from the HOD model with assembly bias. In the standard HOD model, the constraints on $\alpha$ are slightly narrower and favor larger values of $\alpha$. The best-fit and the 1-sigma uncertainties of $\alpha$ are 1.17$^{+0.04}_{-0.04}$ and 1.11$^{+0.06}_{-0.06}$ for the standard and the assembly bias model respectively. In the standard model, the constraint on $\mmin$ is narrower and favors smaller values of $\mmin$. The best-fit and the 1-sigma uncertainties of $\mmin$ are 11.99$^{+0.15}_{-0.08}$ and 12.13$^{+0.39}_{-0.18}$ for the standard and the assembly bias model respectively.}
\label{fig:post_wp_20}
\end{center}
\end{figure*}

%%%%%%%%%%%%%%%%%%%%%%%%%%%%%%%%%%%%%%%%%%%%%%%%%%%%%%%%
% gmf-prediction
%%%%%%%%%%%%%%%%%%%%%%%%%%%%%%%%%%%%%%%%%%%%%%%%%%%%%%%%

\begin{figure*}
\includegraphics[width=\textwidth]{paper_model_decgmf.pdf}
\includegraphics[width=\textwidth]{paper_model_hodgmf.pdf}
\caption{We compare the posterior prediction for the observable $\gmf$ to the Berlind+06 measurements in three Luminosity threshold samples: $\mr<-18$, $\mr<-19$, and $\mr<-20$. The darker and the lighter shaded regions represent the 68$\%$ and the 95$\%$ confidence regions of the posterior predictions. $\bf{Top \; panel}$: 1-sigma and 2-sigma posterior prediction of the Heaviside Assembly bias HOD model. $\bf{Bottom \; panel}$: 1-sigma and 2-sigma posterior predictions of the the standard HOD model.}
\label{fig:gmf-predictions}
\end{figure*}

%%%%%%%%%%%%%%%%%%%%%%%%%%%%%%%%%%%%%%%%%%%%%%%%%%%%%%%%
% gmf-occupation-prediction
%%%%%%%%%%%%%%%%%%%%%%%%%%%%%%%%%%%%%%%%%%%%%%%%%%%%%%%%

\begin{figure*}
\begin{center}
\includegraphics[scale=0.5]{paper_abhod_ncengmf.pdf}\\
\includegraphics[scale=0.5]{paper_abhod_nsatgmf.pdf}\\
\includegraphics[scale=0.5]{paper_abvshod_gmf.pdf}

\caption{We present the posterior predictions for the expected number of galaxies as a function of host halo mass in three luminosity threshold samples: $\mr<-18$, $\mr<-19$, and $\mr<-20$. The shaded regions mark the 68 $\%$ confidence region of the posterior predictions. The observables used in fitting are $\bar{n}$ and $\gmf$. $\bf{Top \; panel}$: posterior predictions for the expected number of centrals as a function of host halo mass for the two population of halos: $type$-1 (high-$c$, shown with blue) and $type$-2 (low-$c$, shown with yellow) halos. The expected number of centrals in $type$-1 and $type$-2 halos at a fixed halo mass are in agreement. $\bf{Middle \; panel}$: posterior predictions for the expected number of satellites as a function of host halo mass for the two population of halos. The expected number of satellites in two population of halos at a fixed halo mass are in agreement.$\bf{Bottom \; panel}$: posterior predictions for the expected number of galaxies as a function of host halo mass for two models: the standard HOD (shown in red), and decorated HOD model (shown in blue). The prediction of the two models for the expected number of galaxies as a function of host halo mass are in agreement.}
\label{fig:gmf-hod-predictions}
\end{center}
\end{figure*}


%%%%%%%%%%%%%%%%%%%%%%%%%%%%%%%%%%%%%%%%%%%%%%%%%%%%%%%%
% wp-prediction
%%%%%%%%%%%%%%%%%%%%%%%%%%%%%%%%%%%%%%%%%%%%%%%%%%%%%%%%
\begin{figure*}
\begin{center}
\includegraphics[scale = 0.3]{paper_model_decwp.pdf} \\
\includegraphics[scale = 0.3]{paper_model_hodwp.pdf}
\caption{We compare the posterior prediction for the observable $\wpp$ to the Guo+15 measurements in five Luminosity threshold samples: $\mr<-18$, $\mr<-18.5$, $\mr<-19$, $\mr<-19.5$, and $\mr<-20$. The darker and the lighter shaded regions represent the 68$\%$ and the 95$\%$ confidence regions of the posterior predictions. $\bf{Top\;panel}$: 1-sigma and 2-sigma posterior prediction of the Heaviside Assembly bias HOD model. $\bf{Bottom\;panel}$: 1-sigma and 2-sigma predictions of the the standard HOD model.}
\label{fig:wp-predictions}
\end{center}
\end{figure*}

%%%%%%%%%%%%%%%%%%%%%%%%%%%%%%%%%%%%%%%%%%%%%%%%%%%%%%%%
% wp-occupation-prediction
%%%%%%%%%%%%%%%%%%%%%%%%%%%%%%%%%%%%%%%%%%%%%%%%%%%%%%%%

\begin{figure*}
\begin{center}
\includegraphics[scale=0.45]{paper_abhod_ncenwp.pdf}\\
\includegraphics[scale=0.45]{paper_abhod_nsatwp.pdf}\\
\includegraphics[scale=0.45]{paper_abvshod_wp.pdf}
\caption{We present the posterior predictions for the expected number of galaxies as a function of host halo mass for three luminosity threshold samples: $\mr<-18$, $\mr<-19$, $\mr<-20$. The shaded regions mark the 68 $\%$ confidence region of the posterior predictions. The observables used in fitting are $\bar{n}$ and $\wpp$. $\bf{Top \; panel}$: posterior predictions for $\langle N_{\rm cen}\rangle$ as a function of host halo mass for the two population of halos: $type$-1 (high-$c$, shown with blue) and $type$-2 (low-$c$, shown with yellow) halos. In the $\mr<-18$ sample and at a fixed host halo mass $M_{h}$, the expected number of centrals in the $type$-2 population of halos is larger than the expected number of centrals in the $type$-1 population. In the $\mr<-19$ sample and at a fixed $M_{h}$, $\langle N_{\rm cen}\rangle$ of the two populations are in agreement. In the $\mr<-20$ sample and at a fixed $M_{h}$, $\langle N_{\rm cen}\rangle$ is larger for $type$-1 population of halos. $\bf{Middle \; panel}$: posterior predictions for the expected number of satellites $\langle N_{\rm sat}\rangle$ as a function of host halo mass for the two population of halos. In the $\mr<-18$ sample and at a fixed host halo mass $M_{h}$, $\langle N_{\rm sat}\rangle$ for the the two populations are in agreement. In the $\mr<-19$ sample and at a fixed $M_{h}$, $\langle N_{\rm sat}\rangle$ of the $type$-2 population is enhanced with respect to that of the $type$-1 halos. In the $\mr<-20$ sample and at a fixed $M_{h}$, $\langle N_{\rm sat}\rangle$ of the two populations are largely in agreement. $\bf{Bottom \; panel}$: posterior predictions for the expected total number of galaxies $\langle N_{\rm g}\rangle$ as a function of host halo mass for two models: the standard HOD (shown in red), and decorated HOD model (shown in blue). In the $\mr<-18$ sample and the high mass end, there is slight tension between $\langle N_{\rm g}\rangle$ derived from standard HOD posterior and the decorated HOD posterior. In the $\mr<-19$ sample, the tension at the high mass end persists but it is less significant.  In the $\mr<-20$ sample, there is some tension between the predictions of the two models at the low mass end.}
\label{fig:wp-hod-predictions}
\end{center}
\end{figure*}

\section*{Acknowledgments}
We would like to thank Michael R. Blanton,  
Boris Leidstadt, Geoffrey~S.~Ryan, and the Blanton-Hogg group meeting for insightful discussions. MV was supported by NSF grant AST-1517237. DWH was supported 
by NSF (grants IIS-1124794 and AST-1517237), NASA (grant NNX12AI50G), and the Moore-Sloan 
Data Science Environment at NYU. Computations 
were performed using computational resources at NYU-HPC. We thank Shenglong Wang, the 
administrator of NYU-HPC computational facility, for his consistent and continuous support 
throughout the development of this project. We also thanks Dan Foreman-Mackey for providing some of the computational resources at the early stages of this work. The posterior probability plots in this paper where made using the software corner (\citealt{corner}).

\bibliographystyle{yahapj}
\bibliography{ab}
\end{document}
